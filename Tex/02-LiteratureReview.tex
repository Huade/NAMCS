\section{Literature review}

\section{Background}

In 2009, the US Congress passed the American Recovery and Reinvestment Act (ARRA), which appropriates funds to promote the adoption and use of health information technology (HIT). The American Recovery and Reinvestment Act has set aside \$2 billion which will go towards programs developed by the National Coordinator and Secretary to help health care providers implement HIT and provide technical assistance through various regional centers \citep{hhs_recovery}.

The Centers for Medicare \& Medicaid Services launched the Medicare and Medicaid Electronic Health Care Record (EHR) Incentive Programs after the passage of ARRA of 2009. These programs provide incentive payments to eligible professionals, eligible hospitals, and critical access hospitals (CAHs) as they adopt, implement, upgrade or demonstrate meaningful use of certified EHR technology. In order to receive the EHR stimulus money, the HITECH act (ARRA) requires eligible physicians to show "meaningful use" of an EHR system.

%http://www.cms.gov/Regulations-and-Guidance/Legislation/EHRIncentivePrograms/Meaningful_Use.html
Eligible physicians must attest to demonstrating meaningful use every year to receive an incentive and avoid a Medicare  
%FIX HERE
%\\\\\\\\\\\\
The HHS Office for Civil Rights (OCR) enforces the HIPAA Privacy and Security Rules, which help keep entities covered under HIPAA accountable for the privacy and security of patients' health information, including electronic health record.

A Harris Interactive survey has found that 63 percent of Americans fear their health data will be stolen, decelerating public acceptance of electronic health records \citep{Kaiser2012}.
%\\\\\\\\\\\\\

\subsection{Effect of EHR on health expenditure}

Limited empirical studies estimated the potential net benefits that could arise by adopting health information technologies (HITs), including the EHR at the national level. The RAND Corporation estimated annual net savings to the health care sector from efficiency alone could be \$77 billion or more based surveys, publications, interviews, and an expert-panel review. \citep{Rand2005}. Hillestad et, al. claimed that effective EHR implementation and networking could eventually save more than \$81 billion annually by improving health care efficiency and safety. Savings could be doubled by using health information technology to preventive care and chronic disease management \citep{Hillestad2005}. However, some other researchers do not find the positive cost-saving effect of EHR adoption on national health expenditure. For example, Adler-Milstein et al. found that ambulatory EHR adoption did not impact total cost, although it slowed ambulatory cost growth \citep{Adler-Milstein2013}. Sidorov claimed that much of the literature on EHRs fails to support the primary rationales for using them and it is unlikely that the U.S. health care bill will decline as a result of the EHR alone \citep{Sidorov2006}. There are also researchers suggest the adoption of EHR has negative effect on cost-reduction  \citep{Teufel2012}.

EHR also provides a platform of predictive analysis, saving health care spending by allocating medical resources efficiently. Bates et al. proposed there are unprecedented opportunities to use big data, acquired from EHR, to reduce the costs of health care in the United States \citep{Bates2014}. Roski et al. also pointed out big data has the potential to create significant value in health care by improving outcomes while lower cost \citep{Roski2014}. However, the integration of EHR into predictive analytics is still challenging. Roski et al. also claimed that big data's success in creating value in the health care sector may require changes in current polices to balance the potential societal benefits of big-data approaches and the protection of patients' confidentiality \citep{Roski2014}.

\subsection{Effect of EHR on healthcare efficiency and quality}

The effect of EHR on efficiency is mixed. A systematical literature review suggested that 92 percent of the recent articles on health information technology show measurable benefits emerging from the adoption of health information technology \citep{Buntin2011}, for example, a study found that hospital with more-advanced health IT had fewer complications, lower mortality, and lower costs than their counterparts\citep{amarasingham2009clinical}. Other suggest that simply adopting electronic health records is likely to be insufficient to drive substantial gains in quality or efficiency \citep{DesRoches2010}. 

Time efficiency is one of the possible outcome of EHR adoption. Physicians spent time on patients-interactions and documentation of clinical information. Clinicians hope that an EHR could increase the patient-interaction time, which improves the quality of health care, while reducing documentation time \citep{leung2003incentives}. However, EHR is unlikely to reduce documentation time \citep{poissant2005impact} and the effect of EHR system adoption on time efficiency is mixed and varying among different institutions \citep{Chaudhry2006}.

Another important factor of healthcare efficiency and quality is the likelihood of follow-up health care appointments. Low "kept appointment" rates adversely affected continuity of care and led to inefficient clinic scheduling processes \citep{myers2001strategies}. Although the CMS listed ``Send reminders to patients per patient preference for preventive/follow-up care'' as an objective in measuring meaningful use of EHR system \citep{cmsincentive14}, the evaluation on the effect of EHR on patient follow-up rate is limited.

Patient-centered education, which provided by EHR based system, allows for the patient to better understand their health and make informed lifestyle adjustments. CMS requires eligible physicians to provide patient-specific education resources to more than 10 percent of all unique patients in order to obtain the EHR incentive program funding \citep{healthit05}. Very limited literature evaluated effect of EHR on patient-specific education resources utilization. 

\subsection{Physicians' financial incentives on EHR}
On the micro level, EHR has a mixed effect on cost-saving of physician practices.

Some scholars claimed that the long-term return of adoption of EHR is positive. For example, Wang et al. estimated that a provider could gain \$86,400 net benefit from using an electronic medical record for a 5-year period, resulting in a positive financial return on investment to the health care organization \citep{Wang2003}. Bell and Thornton claimed that based on the size of a health system and the scope of implementation, benefits of HITs for large hospitals can range from \$37M to \$59M over a five-year period in addition to incentive payments \citep{Bell2011}.

However, more researchers argued that physicians have insufficient financial incentive to implement EHR at the first place. Gans et al. surveyed a nationally representative sample of medical group practices and suggested that adoption of EHR is progressing slowly, at least in smaller practices \citep{Gans2005}. Jha et al. found a similar result that on the basis of responses from 63.1\% of hospitals surveyed, only 1.5\% of U.S. hospitals have a comprehensive electronic-records system \citep{Jha2009}. Adler-Milstein et al. found electronic health records will yield revenue gains for some practices and losses for many by using survey data from 49 community practices. Practices are encountering greater-than-expected barriers to adopting an EHR system \citep{Adler-Milstein2012}.

\subsection{Contribution to literature}

Giving the increasing participation in the Medicare and Medicaid Electronic Health Records (EHR) Incentive Programs, and the increased policy interest in controlling health expenditures, the evaluation of the effect of EHR on physician behavior are of interest.

Although the number of health information technology evaluation studies is rapidly increasing, empirically measured behavior data are limited and inconclusive. Some research projected the potential benefit of adoption of EHR with data from surveys, publications, interviews, and expert-panel reviews, however, there are limited research focus on empirical analysis with national wide data. Literature on outcome of adopting EHR, especially effect of EHR on patient-specific health education prescription, is limited. This paper could contribute to the literature with a national-level perspective and evaluate the outcome of EHR adoption on health education, time spent with MD, and returned appointment rate.

Another major limitation of the literature is its generalization. Insufficient reporting of contextual and implementation factors makes it impossible to determine why most health IT implementations are successful but some are not. This paper will consider which factors may contribute to better outcome of EHR adoption. It could help making government incentive programs more efficient by selecting proper physician practices.


