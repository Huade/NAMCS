\section{Literature review}

\subsection{Positive effect of EMR on cost-saving}

Empirical studies estimated the potential net benefits that could arise by adopting health information technologies (HITs), including EMR. The RAND Corporation estimated annual net savings to the health care sector from efficiency alone could be \$77 billion or more based surveys, publications, interviews, and an expert-panel review. \citep{Rand2005}. Hillestad et, al.claimed that effective EMR implementation and networking could eventually save more than \$81 billion annually by improving health care efficiency and safety. Savings could be doubled by using health information technology to preventive care and chronic disease management.\citep{Hillestad2005} 

Many believes that the implementation of national wide electronic health record contributed to Veteran Health Administration's efficiency. Evans et al. analyzed the effectiveness of the implementation of an enterprise-wide EMR system on productivity in the Veteran Health Administration (VHA) and found that VHA has been able to increase its productivity by nearly 6 percent per year after successful national EMR implementation in 1999 \citep{Evans2005}. 

Some scholars conducted comprehensive literature review on the effect of EMR on efficiency. Buntin et al. reviewed the recent literature on health information technology to determine its effect on outcomes, including quality, efficiency, and provider satisfaction. They found that 92 percent of the recent articles on health information technology reached conclusions that were positive overall \citep{Buntin2011}. Chaudhry et al. conducted a systematic review and found that the major efficiency benefit has been decreased utilization of care, while effect on time utilization is mixed \citep{Chaudhry2006}. 

EMR also provides a platform of predictive analysis, saving health care spending by allocating medical resources efficiently. Bates et al. proposed there are unprecedented opportunities to use big data, acquired from EMR, to reduce the costs of health care in the United States \citep{Bates2014}. Roski et al. also pointed out big data has the potential to create significant value in health care by improving outcomes while lower cost \citep{Roski2014}.

\subsection{Natural or negative effect of EMR on cost-saving}

Some scholars argued that the implementation of EMR has natural or negative effect on cost reduction. Adler-Milstein et al. found that ambulatory EMR adoption did not impact total cost, although it slowed ambulatory cost growth \citep{Adler-Milstein2013}. Sidorov claimed that much of the literature on EMRs fails to support the primary rationales for using them and it is unlikely that the U.S. health care bill will decline as a result of the EMR alone \citep{Sidorov2006}. 

The potential for publication bias is always a limitation of reviews. Researchers often do not evaluate the potential for negative effects, and even when identified, negative results are potentially less likely to be published. However, there are researchers found that EMR has negative effect on cost-reduction. For example, Teufel et al. found that electronic medical record associated with an average 7\% greater cost per case in pediatric \citep{Teufel2012}. Simply adopting electronic health records is likely to be insufficient to drive substantial gains in quality and efficiency \citep{DesRoches2010}

The integration of EMR into predictive analytics is still challenge. Roski et al. claimed that big data's success in creating value in the health care sector may require changes in current polices to balance the potential societal benefits of big-data approaches and the protection of patients' confidentiality \citep{Roski2014}.  


\subsection{Physicians' financial incentives on EMR}
On the micro level, EMR has a mixed effect on cost-saving of physician practices.

Some scholars claimed that the long-term return of adoption of EMR is positive. For example, Wang et al. estimated that a provider could gain \$86,400 net benefit from using an electronic medical record for a 5-year period, resulting in a positive financial return on investment to the health care organization \citep{Wang2003}. Bell and Thornton claimed that based on the size of a health system and the scope of implementation, benefits of HITs for large hospitals can range from \$37M to \$59M over a five-year period in addition to incentive payments \citep{Bell2011}.

However, more researchers argued that physicians have insufficient financial incentive to implement EMR at the first place. Gans et al. surveyed a nationally representative sample of medical group practices and suggested that adoption of EMR is progressing slowly, at least in smaller practices \citep{Gans2005}. Jha et al. found a similar result that on the basis of responses from 63.1\% of hospitals surveyed, only 1.5\% of U.S. hospitals have a comprehensive electronic-records system \citep{Jha2009}. Adler-Milstein et al. found electronic health records will yield revenue gains for some practices and losses for many by using survey data from 49 community practices. Practices are encountering greater-than-expected barriers to adopting an EMR system \citep{Adler-Milstein2012}.

\subsection{Contribution to literature}
Although the number of health information technology evaluation studies is rapidly increasing, most evaluations focus on clinical decision support and computerized provider order entry, rather than electronic medical record which Medicare provide financial incentives. In this paper, I focus mainly on the effect of electronic health record on physician behavior, to meet the needs of health policymakers and researchers.

Empirically measured behavior data are limited and inconclusive. Some research projected the potential benefit of adoption of EMR with data from surveys, publications, interviews, and expert-panel reviews, however, there are limited research focus on empirical analysis with national wide data. Literature on outcome of adopting EMR, including prescription drug utilization, time spent with patients, and returned appointments are also limited. This paper could contribute to the literature with a national-level perspective.

Another major limitation of the literature is its generalization. Insufficient reporting of contextual and implementation factors makes it impossible to determine why most health IT implementations are successful but some are not. This paper will consider which factors may contribute to better outcome of EMR adoption. It could help making government incentive programs more efficient by selecting proper physician practices.

\section{Institutional background}
\subsection{Legislation}
\begin{itemize}
\item In 2009, the US Congress passed the American Recovery and Reinvestment Act, which appropriates funds to promote the adoption and use of health information technology (IT). The American Recovery and Reinvestment Act has set aside \$2 billion which will go towards programs developed by the National Coordinator and Secretary to help health care providers implement HIT and provide technical assistance through various regional centers.
\item The Affordable Care Act requires health insurance issuers to submit data on the proportion of premium revenues spent on clinical services and quality improvement, also known as the Medical Loss Ratio (MLR). MLR requires insurance companies to spend at least 80\% or 85\% of premium dollars on medical care, with the review provisions imposing tighter limits on health insurance rate increases. The cost of enhancing health information technology in a way that improves the quality contributes to the nominator of MLR equation.
\item  The HHS Office for Civil Rights (OCR) enforces the HIPAA Privacy and Security Rules, which help keep entities covered under HIPAA accountable for the privacy and security of patients' health information, including electronic health record.
\end{itemize}

\subsection{Government Programs}
\begin{itemize}
\item The Medicare and Medicaid Electronic Health Care Record (EMR) Incentive Programs provide incentive payments to eligible professionals, eligible hospitals, and critical access hospitals (CAHs) as they adopt, implement, upgrade or demonstrate meaningful use of certified EMR technology.
\item If Medicare eligible professionals, or EPs, do not adopt and successfully demonstrate meaningful use of a certified electronic health record (EMR) technology by 2015, the EPs Medicare physician fee schedule amount for covered professional services will be adjusted down by 1\% each year.
\end{itemize}

\subsection{Public Opinion}
A Harris Interactive survey has found that 63 percent of Americans fear their health data will be stolen, decelerating public acceptance of electronic health records \citep{Kaiser2012}.

