\subsection{Data source}
The data source for this study was National Ambulatory Medical Care Survey (NAMCS) public use micro-data files. NAMCS is a national probability sample survey of visits to office-based physicians conducted by the National Center for Health Statistics, Centers for Disease Control and Prevention. NAMCS has information at visit level, including whether the physician practice has Electronic Medical Record (EMR) system, health education prescription, the breakdown of patients by different payment type, time spent with physician for each visit, and whether the visit is a returned appointment, etc. The sample size for 2008, 2009, and 2010 public use micro-data files, which includes information about adopting EMR, are 28,741, 32,281, and 31,229, respectively.

I used information on adoption of EMR system to identify the treatment groups and potential comparison groups. The survey question was described as "Does your practice use an electronic medical record or health record (EMR/HER) system? (Not including billing records system)." \citep{NAMCSDOC2010}. Three possible groups in this treatment variable including "Yes, all electronic", "Yes, part paper and part electronic", and "No". The other characteristics were used as covariates in the propensity score estimation and models.

The sampling of NAMCS is a multistage process. The first-stage sample includes 112 primary sampling units (PSUs) by geological distribution. The second stage stratified physicians into 15 groups and select physicians within each PSU. The final stage is the selection of patient visits within the annual practices of sample physicians. The basic sampling unit for the NAMCS is the physician-patient encounter or visit. 

Start from 2005, NAMCS includes provider weight that allow researchers to produce aggregated visit statistics at the physician level. In this analysis, I summarized visits level data to physician level data based on recommendation provided by Ambulatory Statistics Branch of Centers for Disease Control and Prevention \citep{SasProcedure}. There are 3,777 physicians' information available after the aggregation. 157 cases were dropped afterward due to incompleteness and 1 case were ignored due to negative physician weight. 3619 observations were available for further analysis. 


