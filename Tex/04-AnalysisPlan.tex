\section{Analysis Plan}
\subsection{Estimating treatment effect with observational data}
Ideally, we would observe physician in three possible conditions: one in which she has fully adopted the EHR system, one in which she has partially adopted the EHR system, and one in which she has not. We can express our evaluation problem as follows: Let $W_i = 1$ for physician $i$ who has fully adopted the EHR system, let $W_i = 2$ for physician $i$ who has partially adopted the EHR system, and let $W_i = 0$ for physician $i$ who has not yet adopted the ENR system. Let $Y_i(1)$ refer to the time efficiency for physician $i$ who has fully adopted the EHR system, let $Y_i(2)$ refer to the time efficiency for physician $i$ who has partially adopted the EHR system, and let $Y_i(0)$ refer to the patient interaction time for physician $i$ who has not adopted the EHR system. Although all outcomes are possible in theory, we cannot observe all possible outcome $Y_i(0)$, $Y_i(1)$, and $Y_i(2)$ for physician $i$ while holding all other conditions constant. We only observe $Y_i(0)$ if $W_i = 0$, $Y_i(1)$ if $W_i = 1$, and $Y_i(2)$ if $W_i = 2$ with our data \citep{imbens2008recent}. People in "treatments" and "control" groups likely different in both observed and unobserved ways.

\subsection{Assumption of causal inference}
There are two assumptions associate with estimating treatment effect. The first assumption is the stable unit treatment value assumption (SUTVA). The SUTVA requires that there is no interference between units, that is, treatment assignment of one unit does not affect potential outcomes of another unit. We cannot test this statistically with our data. However, we can safely assume that this assumption meets in our analysis since there are no evidence to show the EHR adoption at one physician practice has interactions with the outcome in another physician's practice.

The second assumption is no unmeasured confounders. An estimate of the EHR's effect on doctors' behavior relies on an assumption of no unmeasured confounders of treatment assignment, that is, $W_i \prep (Y_i(0),Y_i(1),Y_i(2))$ \citep{imbens2008recent}. In other words, the assignment of study participants to treatment conditions (i.e. fully adopted EHR, partially adopted EHR, and no adoption) is independent of the outcome of these three groups. In experimental settings, treatment groups (in this case, physicians who partially or fully adopted the EHR system) and control group were random assigned, which ensure that both observed and unobserved factors of treatment and control group have similar distribution. If the assignment to adopt the EHR system is based on randomization, this assumption is easy to statisfy and the causal inference would be straightforward. However, this assumption often violates in non-experimental setting. This is a strong assumption with the evaluation of EHR effect since a national level experiment on the effectiveness of the EHR adoption is expensive and infeasible. Violation of unconfoundedness could bias result because of omitted variable bias.

To estimate the effect of the EHR adoption on physician behavior, we can obtain the following model:

\begin{equation*}
Y_{i} = \beta_0 + \beta_1 W_i + \Sigma^k_{i=2} \beta_k X_{ik} + \epsilon_{i}
\end{equation*}

In this model, $Y_{i}$ is the outcome of interest for physician $i$, including percentage rate of patient-specific education resource prescribed, time spent with the physician, and percentage rate of returned patients. $W_i$ is the EHR adoption status for physician $i$, including fully adopted EHR, partially adopted EHR, and no EHR adoption. $X_{ik}$ are $k$ observable characteristics for physician $i$, including MSA status, physician specialty, Solo status, etc. We will describe more details in descriptive statistics section. Coefficient $\beta_1$ estimate the treatment effect of the EHR adoption on three outcome variables if the model is correct and satisfies the assumption of unconfoundedness. 

This condition is unlikely with NAMCS data. For example, physicians in the treatment group A, which they fully adopted the EHR systems, may systematically different than physicians in the control group. This difference could in both observed and unobserved ways. With large number of covariates that has unknown functional relationship with treatment and outcome, it is hard to specify regression adjustment model. Without appropriate instrumental variable or regression discontinuity cutoff available, the propensity score matching method is one of few available techniques that can be used to access the treatment effect of the EHR system on physician behavior.

%WHY PS
\subsection{Propensity score estimation}
As described above, estimating causal effects with observational data is challenging since it involves estimating the unobserved potential outcomes. Propensity score methods attempt to replicate two features of randomized experiments. On the one hand, propensity score methodologies can create groups that look only randomly different from one another (at least on observed variables). On the other hand, propensity score methods do not use outcome variables when setting up the design. With these two features, treatment assignment and the observed covariates are conditionally independent given the propensity score \citep{guo2014propensity}:

\begin{equation*}
\textbf{X_i} \prep W_i | e(X_i)
\end{equation*}

Conditional on the propensity score, each physician has the same probability of assignment to treatment, as in a randomized experiment setting. After propensity score estimation, physicians in the control group who have not adopted the EHR system are comparable with those who in treatment groups with similar propensity scores, at least on observable characteristics. 

Hirano et al. claimed that the resulting estimate is asymptotically efficient if the propensity score is estimated non-parametrically using a series estimator \citep{hirano2003efficient}. In this analysis, we used Generalized Boosted Machine (GBM) model \citep{mccaffrey2004propensity} to estimate the propensity score of each physician. GBM is a general, automated, data-adaptive algorithm that fits several models by way of a regression tree, and then merges the predictions produced by each model. It can use all available covariates and is not subject to the particular modeling choices made by the analyst \citep{hillm2015short}. Comparing with traditional models, GBM model offer numerous advantage to solve the variable specification problem \citep{guo2009propensity}. Our boosted model uses the default setting of twang package \citep{mccaffrey2013tutorial} with R \citep{rbase}, which has 10,000 GBM interactions, three interactions, a bagging fraction of 1.0, and a shrinkage parameter of 0.01, based on McCaffrey's (2013) recommendation. We use physician weight as sample weight in multinomial propensity score estimation procedure.

To assess the quality of propensity score estimation, we use diagnostics to check the balance after propensity score weighting. The goal of propensity score estimation and weighting is to have similar covariate distributions in the matched treated, and control groups. We use both numerical summaries of balance and graphic summaries of balance to evaluate the quality of propensity score weighting. We relied primarily on the absolute standardized difference (ASD, also referred to as the Effect Size or the absolute standardized mean difference) to assess the balance after weighting. 

\subsection{Propensity score weighted regression model}
The essential feature of propensity score weighting model is the treatment of estimated propensity scores as sampling weights to perform a weighted outcome analysis. The control of selection biases is achieved through weighting. Counterfactuals are estimated through a regression model \citep{guo2009propensity}.

When dimension of pre-treatment variables $\textbf{X}$ is large, it is difficult to ensure both the regression model is correct, and a consistent estimator will be obtained \citep{rubin1997estimating}. Also, the estimated modeling leads to extrapolation if the distribution of some confounders do not overlap with each other, since the effect is primarily determined by treated subjects in one region of $\textbf{X}$ space and by control subjects in another. In contrast, the regression model with propensity score weighting largely circumvents this since pretreatment variables $\textbf{X}$ and treatment group variable $W$ should be approximately independent after propensity score estimation. By adding covariates into the regression adjustment, we will obtain ``double robustness'' that further improve the precision of estimators \citep{lunceford2004stratification}. We used an estimate of the propensity score as weights, and uses these weights in a weighted regression of the potential outcome on treatment and observed covariates.

We estimate a separate propensity score weighted regression model for each outcome. We include covariates that have maximum ASD greater than 0.1. 

\subsection{Sensitivity tests}

Finally, we conduct sensitivity tests of the following four cases.

First, we test the robustness of the result to different covariate controls with multinomial propensity score weighted regression models. We test this in two cases: (1) including only treatment variable with no covariates; (2) including all possible covariates and treatment assignment variable.

Second, we test whether the results are robust to different multinomial propensity score weighted generalized regression model. Based on the distribution of dependent variables, we use Binomial regression for the EHR adoption status on health education prescription rate, Poisson regression for the EHR adoption status on time spent with MD, and Binomial regression for the EHR adoption status on returned appointment rate.

Third, we test whether the results are robust to propensity score weighted binary treatment assignment. We create two separated datasets. One with only physicians who have fully adopted EHR and control group. Another with only physicians who have partially adopted EHR system and control group. We estimated propensity score with binary treatment and estimate the effect of full EHR adoption and partial EHR adoption on outcome variables.

Fourth, we test the robustness of the result with propensity score matching approach. We use nearest neighbor matching for binary treatment cases and assess the treatment effect of the EHR adoption.

