\section{Analysis Plan}
Ideally, we would observe physician in three possible conditions: one in which she has fully adopted the EHR system, one in which she has partially adopted the EHR system, and one in which she has not. We can express our evaluation problem as follows: Let $W_i = 1$ for physician $i$ who has fully adopted the EHR system, let $W_i = 2$ for physician $i$ who has partially adopted the EHR system, and let $W_i = 0$ for physician $i$ who has not yet adopted the ENR system. Let $Y_i(1)$ refer to the time efficiency for physician $i$ who has fully adopted the EHR system, let $Y_i(2)$ refer to the time efficiency for physician $i$ who has partially adopted the EHR system, and let $Y_i(0)$ refer to the patient-interaction time for physician $i$ who has not not adopted the EHR system. Although all outcomes are possible in theory, we cannot observe all possible outcome $Y_i(0)$, $Y_i(1)$, and $Y_i(2)$ for physician $i$ while holding all other conditions constant. We only observe $Y_i(0)$ if $W_i = 0$, $Y_i(1)$ if $W_i = 1$, and $Y_i(2)$ if $W_i = 2$ with our data \citep{imbens2008recent}.

% http://tex.stackexchange.com/questions/79434/double-perpendicular-symbol-for-independence
In experimental setting, treatment group (in this case, physicians who partially or fully adopted the EHR system) and control group were random assigned, which ensure that both observed and unobserved factors of treatment and control group have similar distribution. If assignment to adopt the EHR system is based on randomization, this causal inference would be straightforward. However, a national level experiment on the effectiveness of the EHR adoption is expensive and infeasible. Because of that, an estimate of the EHR's effect on physicians' behavior relies on an assumption of no unmeasured confounders of treatment assignment, that is, $W_i \independent (Y_i(0),Y_i(1),Y_i(2))$ \citep{imbens2008recent}. In other words, the assignment of study participants to treatment conditions (i.e. fully adopted EHR, partially adopted EHR, and no adoption) is independent of the outcome of these three groups. This assumption often violates in non-experimental setting. Violation of unconfoundedness could bias results because of omitted variable bias.

%WHY PS
Estimating causal effects with observational data is challenging since it involves estimating the unobserved potential outcomes. Physicians in treatment group A, which they fully adopted the EHR systems, may systematically different than physicians in the control group. This difference could in both observed and unobserved ways. With large number of covariates that has unknown functional relationship with treatment and outcome, it is difficult to specify regression adjustment model. Without appropriate instrumental variable or regression discontinuity cutoff available, the propensity score matching method is one of few available techniques that can be used to access the treatment effect of the EHR system on physician behavior.


With extensive information available in NAMCS data, I added various pre-treatment covariates with physician weight when estimating propensity score.  

%WHY Weighting
When dimision of pre-treatment variables $\textbf{X}$ is large, it is difficult to ensure both the regression model is correct and a consistent estimator will be obtained \citep{rubin1997estimating}. Also, the estimated modeling leads to extrapolation if the distribution of some confounders do not overlap with each other, since the effect is primarily determined by treated subjects in one region of $\textbf{X}$ space and by control subjects in another. In contrast, the regression model with propensity score weighting largely circumvents this since pretreatment variables $\textbf{X}$ and treatment group variable $W$ should be approximately independent after propensity score estimation. By adding covariates into the regression adjustment, we will obtain ``double robustness'' which further improve the precision of estimators \citep{lunceford2004stratification}. I used an estimate of the propensity score as weights, and uses these weights in a weighted regression of the outcome on treatment and covariates. 

% WHY GBM
Hirano et al. claimed that the resulting estimate is asymptotically effcient if the propensity score is estimated non-parametrically using a series estimator \citep{hirano2003efficient}. I used Generalized Boosted Machine (GBM) model to estimate propensity score of each physicians. GBM is a general, automated, data-adaptive algorithm that fits several models by way of a regression tree, and then merges the predictions produced by each model. Comparing with traditional models, GBM model offer numerous advantage to solve the variable specification problem \citep{guo2009propensity}. 
