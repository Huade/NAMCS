\section{Analysis Plan}
Ideally, we would observe physician in three possible conditions: one in which she has fully adopted the EHR system, one in which she has partially adopted the EHR system, and one in which she has not. We can express our evaluation problem as follows: Let $W_i = 1$ for physician $i$ who has fully adopted the EHR system, let $W_i = 2$ for physician $i$ who has partially adopted the EHR system, and let $W_i = 0$ for physician $i$ who has not yet adopted the ENR system. Let $Y_i(1)$ refer to the time efficiency for physician $i$ who has fully adopted the EHR system, let $Y_i(2)$ refer to the time efficiency for physician $i$ who has partially adopted the EHR system, and let $Y_i(0)$ refer to the patient-interaction time for physician $i$ who has not not adopted the EHR system. Although all outcomes are possible in theory, we cannot observe all possible outcome $Y_i(0)$, $Y_i(1)$, and $Y_i(2)$ for physician $i$ while holding all other conditions constant. We only observe $Y_i(0)$ if $W_i = 0$, $Y_i(1)$ if $W_i = 1$, and $Y_i(2)$ if $W_i = 2$ with our data \citep{imbens2008recent}.

In experimental setting, treatment group (in this case, physicians who partially or fully adopted the EHR system) and control group were random assigned, which ensure that both observed and unobserved factors of treatment and control group have similar distribution. A national level experiment on the effectiveness of the EHR adoption is expensive and infeasible. Because of that, an estimate of the EHR's effect on physicians' behavior relies on an assumption of no unmeasured confounders. Violation of unconfoundedness could bias results because of omitted variable bias.

%WHY PS
Estimating causal effects with observational data is challenging since it involves estimating the unobserved potential outcomes. Physicians in treatment group A, which they fully adopted the EHR systems, may systematically different than physicians in the control group. This difference could in both observed and unobserved ways. Without appropriate instrumental variable or regression discontinuity cutoff available, they propensity score matching method is one of few available techniques that can be used to access the treatment effect of the EHR system on physician behavior.

Propensity score rely on assumption of no unmeasured confounders. It is possible that confounding variables are omitted inadvertently. With extensive information available in NAMCS data, I added various covariates when estimating propensity score. Knowing that many factors contribute to the decision of whether physician adopt the EHR or not, I included practice size, physician specialty, geological location, and more. 

%WHY Weighting
I used an estimate of the propensity score as weights, and uses these weights in a weighted regression of the outcome on treatment and covariates.

% WHY GBM
I used Generalized Boosted Machine (GBM) model to estimate propensity score of each physicians. GBM is a general, automated, data-adaptive algorithm that fits several models by way of a regression tree, and then merges the predictions produced by each model. Comparing with traditional models, GBM model offer numerous advantage to solve the variable specification problem \citep{guo2009propensity}. 
