As shown in Table \ref{tab.1}, practices that adopt EMR system has statistically significant different than their counterparts.

The adoption of EMR has statistically significant difference between different practice ownership ($\chi^2=18.33$). Physicians or physician group has lower likelihood to adopt EMR system. Health Maintenance Organization, community health center, or other health care corporations tends to have higher likelihood to adopt EMR system. The difference between other health care organizations are not substantially significant. In addition, Solo practice has statistically significant less likelihood to adopt EMR system ($\chi^2=-9.74$).

Most practices inside NAMCS data are located in Metropolitan Statistical Area. There are no statistically significant relationship between adoption of EMR system and whether the practice is in metropolitan statistical areas ($\chi^2=0.02$).

Practice size, measured by number of managed care contracts, has statistically significant relationship with the adoption of EMR system ($\chi^2=10.83$). Larger practices have higher likelihood to adopt EMR system, while practice with less than 10 managed care contracts have less likelihood to adopt EMR system.

Physician specialty has statistically significant relationship with the adoption of EMR system as well ($\chi^2=7.24$). Physicians specialized in general and family practice and internal medicine have higher likelihood to adopt EMR system, while physicians specialized in psychiatry, ophthalmology, and general surgery has less likelihood to adopt EMR system.

Geographic regions also has statistically significant relationship with the adoption of EMR system ($\chi^2=3.87$). Physicians that are in West region has higher likelihood to adopt EMR system, while physicians in Northeast or South region have less likelihood to adopt it. There are no significant difference among physicians in Midwest region.
