\section{Descriptive Statistics}

As shown in Table \ref{tab:descriptive.1}, more physician practice fully adopted the EHR system since 2008. While 54.21\% of physicians reported that they have no EHR system adoption in 2008, 5.75\% percentage points less physician report they have no EHR adoption in 2010, reducing 10.6\% percent comparing to year 2008. Meanwhile, 38.8\% physicians reported they have full EHR system adoption on 2010, while only 27.49\% physicians reported then have fully adopted the EHR system in 2008. Comparing with year 2008 (18.3\%), less physicians partially adopt the EHR system in 2010 (12.73\%). The result suggest the adoption rate of the EHR system is growing rapidly after the implementation of the EHR incentive program.

The adoption of EHR has statistically significant difference between different practice ownerships ($p<0.0001$). Physicians or physician group has lower likelihood to adopt EHR system. In our sample, 53.95\% respondents who are physicians or physician groups reported they have no EHR adoption. Health Maintenance Organization (80.51\%) has the highest likelihood to fully adopt the EHR system, among all health care practice ownerships. There are no substantially difference of partially adopt the EHR system between different practice ownership types. The fully adoption rate among other hospital (35.26\%), other health care corporations (47.59\%), or all others (44.14\%) are also variance. 

There are no statistically significant relationship between adoption of the EHR system and whether the practice is in metropolitan statistical areas ($\chi^2=1.4319$). The fully adoption rate of MSA area (32.93\%) and non-MSA area (33.08\%) is close to the national average (32.94\%).

Another physician practice's characteristics of interest is managed care contracts. The contract between a physician and a managed care organization can affect payment, office organization, practices and procedures, and confidential records as well as clinical decision making \citep{mcc2008}.

Physician specialty has statistically significant relationship with the adoption of EHR system as well ($\chi^2=7.24$). Physicians specialized in general and family practice and internal medicine have higher likelihood to adopt EHR system, while physicians specialized in psychiatry, ophthalmology, and general surgery has less likelihood to adopt EHR system.

Geographic regions also has statistically significant relationship with the adoption of EHR system ($\chi^2=3.87$). Physicians that are in West region has higher likelihood to adopt EHR system, while physicians in Northeast or South region have less likelihood to adopt it. There are no significant difference among physicians in Midwest region.
