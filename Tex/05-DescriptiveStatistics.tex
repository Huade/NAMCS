\section{Descriptive Statistics}

As shown in Table \ref{tab:descriptive.1}, more physician practice fully adopted the EHR system since 2008. While 35.2\% of physicians reported that they have no EHR system adoption in 2008, 3.6\% percentage points less physician report they have no EHR adoption in 2010, reducing 10.22\% percent comparing to year 2008. Meanwhile, 38.8\% physicians reported they have full EHR system adoption on 2010, increasing 42.12\% comparing with year 2008. Less physicians partially adopt the EHR system in 2010, comparing with year 2008. The result suggest the adoption rate of the EHR system is growing rapidly after the implementation of the EHR incentive program.


and Table \ref{tab:descriptive.2}, practices that adopt EHR system has statistically significant different than their counterparts.

The adoption of EHR has statistically significant difference between different practice ownership ($\chi^2=18.33$). Physicians or physician group has lower likelihood to adopt EHR system. Health Maintenance Organization, community health center, or other health care corporations tends to have higher likelihood to adopt EHR system. The difference between other health care organizations are not substantially significant. In addition, Solo practice has statistically significant less likelihood to adopt EHR system ($\chi^2=-9.74$).

Most practices inside NAMCS data are located in Metropolitan Statistical Area. There are no statistically significant relationship between adoption of EHR system and whether the practice is in metropolitan statistical areas ($\chi^2=0.02$).

Practice size, measured by number of managed care contracts, has statistically significant relationship with the adoption of EHR system ($\chi^2=10.83$). Larger practices have higher likelihood to adopt EHR system, while practice with less than 10 managed care contracts have less likelihood to adopt EHR system.

Physician specialty has statistically significant relationship with the adoption of EHR system as well ($\chi^2=7.24$). Physicians specialized in general and family practice and internal medicine have higher likelihood to adopt EHR system, while physicians specialized in psychiatry, ophthalmology, and general surgery has less likelihood to adopt EHR system.

Geographic regions also has statistically significant relationship with the adoption of EHR system ($\chi^2=3.87$). Physicians that are in West region has higher likelihood to adopt EHR system, while physicians in Northeast or South region have less likelihood to adopt it. There are no significant difference among physicians in Midwest region.
