\chapter{Descriptive Statistics}

As shown in Table \ref{tab:descriptive.1}, more physician practice fully adopted the EHR system since 2008. While 54.21\% of physicians reported that they have no EHR system adoption in 2008, 5.75\% percentage points less physician report they have no EHR adoption in 2010, reducing 10.6\% percent comparing to year 2008. Meanwhile, 38.8\% physicians reported they have full EHR system adoption on 2010, while only 27.49\% physicians reported then have fully adopted the EHR system in 2008. Comparing with year 2008 (18.3\%), fewer physicians partially adopt the EHR system in 2010 (12.73\%). The result suggests the adoption rate of the EHR system is growing rapidly after the implementation of the EHR incentive program.

The adoption of EHR has statistically significant difference between different practice ownerships ($p<0.0001$). Physicians or physician group has lower likelihood to adopt EHR system. In our sample, 53.95\% respondents who are physicians or physician groups reported they have no EHR adoption. Health Maintenance Organization (80.51\%) has the highest likelihood to fully adopt the EHR system, among all health care practice ownerships. There is no substantial difference of partially adopt the EHR system between different practice ownership types. The full adoption rate among other hospital (35.26\%), other health care corporations (47.59\%), or all others (44.14\%) are also variance. 

There are no statistically significant relationship between adoption of the EHR system and whether the practice is in metropolitan statistical areas ($\chi^2=1.4319$). The full adoption rate of MSA area (32.93\%) and non-MSA area (33.08\%) is close to the national average (32.94\%). However, geographic regions have statistically significant relationship with the adoption of the EHR system ($\chi^2=16.41$). Physicians that are in West region has higher likelihood to adopt EHR system, while physicians in Northeast or South region have less likelihood to adopt it, comparing with physicians who are in Midwest region.

Another physician practice's characteristics of interest are the number of managed care contracts. The contract between a physician and a managed care organization can affect payment, office organization, practices and procedures, and confidential records as well as clinical decision-making \citep{mcc2008}. In general, practice with higher number of managed tends to have higher adoption rate of the EHR system. 38.08\% physician practice with more than ten managed contracts has fully adopted the EHR system while only 20.92\% physician practice with no managed contracts fully adopted the EHR system. There are no statistically or substantially difference of partially adoption rate among difference managed care contracts.

Physician specialty has statistically significant relationship with the adoption of the EHR system ($p < 0.0001$). Among all physician specialties, general and family practice have the highest likelihood to fully adopt the EHR system. 43.17\% physicians who are general or family practice reported they had fully adopted the EHR system. Ophthalmologists have the lowest likelihood to adopt the EHR system. More than half of ophthalmologists reported they have no EHR adoption. Among all other physician specialities, oncology (38.36\%), internal medicine (37.76\%), urology (37.57\%), and orthopedic surgery (37.09\%) also have higher likelihood of fully EHR adoption.
 
Comparing with the group practice, solo practice has less likelihood to fully adopt the EHR system. Over half of group practice fully or partially adopted the EHR system, while less than 40\% solo practices adopted the EHR system. 

Physician practices with difference EHR adoption status tend to have different payment structure. Practice with fully EHR adoption tends to have higher percentage of privately insured patients. On average, 60.85\% visits are privately insured patients at a practice without the EHR system adoption while 64.75\% visits are privately insured patients at the practice with the EHR system fully adopted. There is slight difference among Medicare, Medicaid, or self-paid patients between different EHR adoption status. Practice with the EHR system fully adopted has slightly less likelihood to accept Medicare, Medicaid, or self-paid patient, comparing with the control group. Practices with fully adopted EHR system have higher likelihood to accept work compensation patients.

As for the characteristic of patients, there is no significant age difference between the fully treated group and the control group. Practice with the EHR system partially adopted has higher average patient age. This is constant with patients insurance status since practices with partially adopted EHR system has higher likelihood to accept Medicare patients.  There is significant difference between the average number of chronological disease between the fully treated group and the control group. Less patient with complex chronological conditions visited practice without the EHR system. 


The adoption status of the EHR system has potential influence on outcome variables. Physicians who have fully or partially adopted the EHR system have higher likelihood to prescribe patient-specific education resource. While 39.7\% patients have received education resources during their visit at practice without the EHR system, more than 44\% patients received education resource during their visit at practice with the EHR system adoption. As for patient-physician interaction time, there is no systematical difference between the practice with the EHR adoption or not. On average, patients spend 22 minutes with their medical doctor, and there is no substantial difference among different EHR adoption status. For returned appointment rate, physician practice who have fully adopted the EHR system has lower returned appointment rate (66.16\%), comparing with the other groups whose returned appointment rate is higher than 71 percent.


